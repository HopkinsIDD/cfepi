
%% Author_tex.tex
%% V1.0
%% 2012/13/12
%% developed by Techset
%%
%% This file describes the coding for rsproca.cls
% \documentclass[PTRSB]{rsos}
\documentclass[PTRSB]{rsos}
% \makeatletter\if@twocolumn\PassOptionsToPackage{switch}{lineno}\else\fi\makeatother
% 
% %Publisher: Royal Society
% %Template Provided By: Typeset
% 
% \usepackage{amsmath,tabulary,graphicx}
% \usepackage[utf8]{inputenc}
% \long\def\ack#1{{\vskip5.5pt\noindent \fontsize{8}{11}\selectfont#1}}
% 
% %%%%%%%%%%%%%%%%%%%%%%%%%%%%%%%%%%%%%%%%%%%%%%%%%%%%%%%%%%%%%%%%%%%%%%%%%%
% % Following additional macros are required to function some 
% % functions which are not available in the class used.
% %%%%%%%%%%%%%%%%%%%%%%%%%%%%%%%%%%%%%%%%%%%%%%%%%%%%%%%%%%%%%%%%%%%%%%%%%%
% \usepackage{url,multirow,morefloats,floatflt,cancel,tfrupee}
% \makeatletter
% 
% 
% \AtBeginDocument{\@ifpackageloaded{textcomp}{}{\usepackage{textcomp}}}
% \makeatother
% \usepackage{colortbl}
% \usepackage{xcolor}
% \usepackage{pifont}
% \usepackage[nointegrals]{wasysym}
% \urlstyle{rm}
% \makeatletter
% 
% %%%For Table column width calculation.
% \def\mcWidth#1{\csname TY@F#1\endcsname+\tabcolsep}
% 
% %%Hacking center and right align for table
% \def\cAlignHack{\rightskip\@flushglue\leftskip\@flushglue\parindent\z@\parfillskip\z@skip}
% \def\rAlignHack{\rightskip\z@skip\leftskip\@flushglue \parindent\z@\parfillskip\z@skip}
% 
% 
% %\if@twocolumn\usepackage{dblfloatfix}\fi
% \usepackage{ifxetex}
% \ifxetex\else\if@twocolumn\usepackage{dblfloatfix}\fi\fi
% 
% \AtBeginDocument{
% \expandafter\ifx\csname eqalign\endcsname\relax
% \def\eqalign#1{\null\vcenter{\def\\{\cr}\openup\jot\m@th
%   \ialign{\strut$\displaystyle{##}$\hfil&$\displaystyle{{}##}$\hfil
%       \crcr#1\crcr}}\,}
% \fi
% }
% 
% %For fixing hardfail when unicode letters appear inside table with endfloat
% \AtBeginDocument{%
%   \@ifpackageloaded{endfloat}%
%    {\renewcommand\efloat@iwrite[1]{\immediate\expandafter\protected@write\csname efloat@post#1\endcsname{}}}{}%
% }%
% 
% \def\BreakURLText#1{\@tfor\brk@tempa:=#1\do{\brk@tempa\hskip0pt}}
% \let\lt=<
% \let\gt=>
% \def\processVert{\ifmmode|\else\textbar\fi}
% \let\processvert\processVert
% 
% \@ifundefined{subparagraph}{
% \def\subparagraph{\@startsection{paragraph}{5}{2\parindent}{0ex plus 0.1ex minus 0.1ex}%
% {0ex}{\normalfont\small\itshape}}%
% }{}
% 
% % These are now gobbled, so won't appear in the PDF.
% \newcommand\role[1]{\unskip}
% \newcommand\aucollab[1]{\unskip}
%   
% \@ifundefined{tsGraphicsScaleX}{\gdef\tsGraphicsScaleX{1}}{}
% \@ifundefined{tsGraphicsScaleY}{\gdef\tsGraphicsScaleY{.9}}{}
% % To automatically resize figures to fit inside the text area
% \def\checkGraphicsWidth{\ifdim\Gin@nat@width>\linewidth
% 	\tsGraphicsScaleX\linewidth\else\Gin@nat@width\fi}
% 
% \def\checkGraphicsHeight{\ifdim\Gin@nat@height>.9\textheight
% 	\tsGraphicsScaleY\textheight\else\Gin@nat@height\fi}
% 
% \def\fixFloatSize#1{}%\@ifundefined{processdelayedfloats}{\setbox0=\hbox{\includegraphics{#1}}\ifnum\wd0<\columnwidth\relax\renewenvironment{figure*}{\begin{figure}}{\end{figure}}\fi}{}}
% \let\ts@includegraphics\includegraphics
% 
% \def\inlinegraphic[#1]#2{{\edef\@tempa{#1}\edef\baseline@shift{\ifx\@tempa\@empty0\else#1\fi}\edef\tempZ{\the\numexpr(\numexpr(\baseline@shift*\f@size/100))}\protect\raisebox{\tempZ pt}{\ts@includegraphics{#2}}}}
% 
% %\renewcommand{\includegraphics}[1]{\ts@includegraphics[width=\checkGraphicsWidth]{#1}}
% \AtBeginDocument{\def\includegraphics{\@ifnextchar[{\ts@includegraphics}{\ts@includegraphics[width=\checkGraphicsWidth,height=\checkGraphicsHeight,keepaspectratio]}}}
% 
% \def\URL#1#2{\@ifundefined{href}{#2}{\href{#1}{#2}}}
% 
% %%For url break
% \def\UrlOrds{\do\*\do\-\do\~\do\'\do\"\do\-}%
% \g@addto@macro{\UrlBreaks}{\UrlOrds}
% 
% \@ifundefined{quoteAttrib}
% 	{\long\def\quoteAttrib#1{\par\raggedleft\itshape#1\unskip}}
% 	{}
% 
% \@ifundefined{titlequoteAttrib}
% 	{\long\def\titlequoteAttrib#1{\list{}{\topsep-3pt\leftmargin.5in\rightmargin0pt}%
%   \item\relax---\upshape#1\endlist}}{}
% 
% \renewenvironment{quote}
% 	{\list{}{\leftmargin.5in\rightmargin\leftmargin}%
%   \item\relax}
%   {\endlist}
% 
% \newenvironment{title-quote}
% 	{\list{}{\fontsize{10pt}{12pt}\selectfont\leftmargin.5in\itshape\rightmargin\leftmargin}%
%   \item\relax}
%   {\endlist}
% 
% 
% \makeatother
% % \def\floatpagefraction{0.8} 
% % \def\dblfloatpagefraction{0.8}
% \def\style#1#2{#2}
% \def\xxxguillemotleft{\fontencoding{T1}\selectfont\guillemotleft}
% \def\xxxguillemotright{\fontencoding{T1}\selectfont\guillemotright}
% %%%%%%%%%%%%%%%%%%%%%%%%%%%%%%%%%%%%%%%%%%%%%%%%%%%%%%%%%%%%%%%%%%%%%%%%%%
% 
% 
% 
% \Journal{Phil. Trans. R. Soc. B }
% 
% \usepackage[nolists,nomarkers]{endfloat}
% 
% %%%% Article title to be placed here
\title{Single-World Counterfactual Inference}

\author{%%%% Author details
Joshua Kaminsky$^{1}$, Lindsay T. Keegan$^{1}$,\\C. Jessica E. Metcalf$^{12}$ and Justin Lessler$^{1}$}

%%%%%%%%% Insert author address here
\address{$^{1}$Department of Epidemiology, Johns Hopkins Bloomberg School of Public Health, Baltimore, MD, USA\\
$^{2}$Department of Ecology and Evolutionary Biology, Princeton University, Princeton, NJ, USA}

%%%% Subject entries to be placed here %%%%
\subject{Epidemiology, Infectious Disease Modeling}

%%%% Keyword entries to be placed here %%%%
\keywords{Counterfactuals, Infectious disease modeling, Infectious disease Dynamics, Network Modeling}

%%%% Insert corresponding author and its email address}
\corres{Justin Lessler\\
\email{justin@jhu.edu}}

\begin{document}
%%%% Abstract text to be placed here %%%%%%%%%%%%
% Recount words
\begin{abstract} % 154 words MAX 183 words
  Determining the effects of an intervention on a disease outbreak is an important problem in epidemiology.
  However, multiple confounders and sources of noise make it difficult to address. %23 words
  Here we present a method that addresses an important source of noise, stochastic variation within the same population. %21 words
  This allows for improved ability to select the best interventions for an outbreak
  Our method, like many others, employs compartmental models and extends an individual-level effect of the intervention to the population-level effect. %20 words
  We eliminate stochastic variation between epidemics by tracking both what happens with intervention and what happens without intervention within a single simulated epidemic. %24 words
  To simultaneously keep track of both states, we store all transition states that are possible either with or without the intervention on a network. %21 words
  We can actualize this network to recover the events that occur both with and without the intervention. %
  We demonstrate our method on an influenza epidemic and compare the results of our method to an existing method.
  We compare across three interventions: hand-washing, vaccination, antivirals, and a control; and evaluate performance on the estimate of cases averted.
  We find that our method reduces the variance of all estimates, and in one case makes the results statistically significant. %25 words
  Although our method is more computationally intensive than traditional methods, it is still computationally tractable for many common use cases, making it a useful tool for selecting interventions. %17 words
\end{abstract}
%%%%%%%%%%%%%%%%%%%%%%%%%%%


%%%%%%%%%%%%%%% End of first page %%%%%%%%%%%%%%%%%%%%%

\maketitle


\section{Introduction}
Determining how to control an infectious disease outbreak is a fundamental question in epidemiology.
This process often relies on a combination of assessments, expert opinion, and disease surveillance, however in many outbreak settings, these may be unavailable. % CITE THIS PAPER [An approach to and web-based tool for infectious disease outbreak intervention analysis Ashlynn R. Daughton, Nicholas Generous, Reid Priedhorsky & Alina Deshpande]
By using available data across multiple levels, dynamical models provide quantitative estimates of outbreak progression in a timely manner, providing insight into how to respond.
Often these modeling approaches provide the best evidence-based estimates of the impact of an intervention when other methods are not feasible.

Determining the effect of an intervention is fundamentally a counterfactual question: how would the outcome of a given epidemic in a population have been different in the presence of a control strategy versus the absence of that control strategy.
For example, one can ask, in the case of the $2014$ Ebola epidemic in Liberia, how many cases of Ebola would have been averted by the widespread deployment of an effective vaccine.
However, this is impossible to know. 
This inability to know the counterfactual outcome is known as the fundamental problem of causal inference.\cite{holland:1986}

Although when responding to an epidemic we are stuck with the fundamental problem of causal inference, in a modeling framework, this can be averted.
However despite the fact that it can be averted, it is rarely formally addressed. 
While most existing models run separate simulations for both arms of the control regime, it is possible to model two different control strategies in a identical epidemic.
In this paper, we refer to models which run separate simulations for the outbreak and the intervention as multiple-world inference and we present a method here that runs a single simulation for the outbreak and all interventions, which we call single-world inference.

The majority of quantitative efforts used to determine the effectiveness of an intervention employ multiple-world inference \cite{} %CITE LESSLER, RIVERS, LONGINI, AND THE PAPER I ADDED ABOVE
However, multiple-world inference methods suffer from high variance and low statistical power, as the method involves comparing different simulated epidemics which often results in the effect of the intervention being obfuscated by process model variance.
For example, Lessler et al. 2016 encountered a common problem of multiple-world inference while determining the best way to trigger a measles vaccination campaign in Africa. 
They noted the requisite use of aggregate statistics as a limitation resulting from the use of multiple-world inference \cite{lessler-et-al:2016}.
Similarly, Rivers et al. 2014 encountered another common drawback of multiple-world inference when determining the impact of multiple interventions against Ebola.
For one of their control strategies, the variance was too large to detect a significant effect of the intervention \cite{rivers-et-al:2014}.
These papers exemplify the most common problems associated with multiple-world inference methods, however we assume there are far more unpublished examples where statistical power was not sufficient to yield definitive results.

The method we describe here, single-world inference, is an adaptation of a contact-network epidemiological method called percolation \cite{miller-book}.
Importantly, single-world inference avoids the fundamental problem of causal inference by running both arms of the counterfactual (i.e., employing the intervention and doing nothing) within a single simulation.
Percolation models disease transmission as a network whose nodes are people, and whose edges are transmissions (i.e., the transition from susceptible to infected).
It starts with a graph of all possible transmissions and then removes both transmissions that are impossible and transmissions that are possible but do not happen as determined by the stochastic model, this removal of edges is known as ``pruning''.
Any edges that remain after pruning transmit disease.

Single-world inference expands the traditional percolation method in two important ways.
First, it explicitly models time: in single world inference, nodes represent a person at a specific time step and thus it directly models all state changes (e.g., susceptible to infected or infected to recovered) rather than only considering transmissions.
Second, it includes an intervention: in single-world inference, each node gets assigned multiple states (i.e., a state for each intervention), unlike the percolation model in which it is only necessary to assign one state to each node.
As a result, in single-world inference, fewer transitions are removed than percolation would.

Although the vast majority of previous work has applied multiple-world inference to determine the outcome of an intervention on an epidemic, Kenah and Miller 2011 applied a single-world inference method to determine the impact of vaccinating highly connected nodes in a contact network \cite{kenah-miller:2011}.
While Kenah and Miller 2011 also percolate a contact network before applying an intervention, they do not describe a method to apply single-world inference to a general class of models or interventions \cite{kenah-miller:2011}.
Here we outline single-world inference using an influenza outbreak as a guiding example and then present a generalized method that can be applied to a whole class of compartmental models and a diversity of intererventions.

\section{Methods}
Single-world inference can take a large class of epidemiological models to explore an outbreak trajectories with and without intervention. 
For the purpose of this paper, we restrict our development to compartmental models, extensions to other classes of models are beyond the scope of this paper.
Compartmental models consist of classes (or compartments) of people and describe the transitions between the different compartments.
There are many variations on compartmental models, but we focus on models where the transitions are stochastic.
Further, compartmental models generally describe population-level events, however most population-level models can be adapted to be written as individual-level models.
In this paper, we use individual-level compartmental models because often the only information we have on an intervention is at the individual-level.

\subsection{Example: Influenza}
To demonstrate single-world inference, we construct an influenza example using a standard Susceptible, Infected, Removed compartmental model.
By modeling influenza transmission within a single season we do not account for loss of immunity and we assume that births and deaths are negligible.
We allow the probability of an infectious contact between any particular susceptible-infected pair to be $\beta = 0.78$ per day and the probability of an infected individual recovering at any time step to be $\gamma = 0.44$ per day \cite{forsberg-white-et-al:2009}.

%% TABLE 1
\begin{table}\label{table:performance}
\caption{Space and time complexity results for a brute-force single-world inference, our single-world inference (first and second prunings), and multiple-world inference.}
\begin{tabular}{|l|l|l|}
  \hline
  Task & Space Use & Time / simulation\\\hline
  Unpruned Network & $58.4$ TB & \textemdash \\\hline
  Prune Network (process model) & $1.73$ GB & 54 seconds \\\hline
  Prune Network (Control Strategy) & $1.3-1.4$ KB & $16-20$ seconds \\\hline
  Multiple-World Inference& $2.6-2.8$ KB &  22-24 seconds\\\hline
\end{tabular}
\end{table}

%% FIGURE 1
\begin{figure}\label{fig:pruning}
\includegraphics[width=\textwidth]{../figures/figure6} %Probably turn the coloration into a legend.
\caption{Diagram demonstrating pruning.  In all panels, the nodes represent people at different times.  The nodes in the same horizontal line represent a single person, moving through time.  Each node is split into three sections, representing the possibility of being susceptible (upper left, purple), infected (upper right, orange), or recovered (bottom, green).  If we know a state to be impossible, we color the corresponding section of that node grey instead of its normal color.  The green edges represent recovery, the orange edges represent infectious contact.  \textbf{a)}--\textbf{i)} show an example of pruning.  The first column represents updating possible initial values.  The second column represents removing impossible events without testing them.  Removed edges are marked with an $X$.  The third column represents testing the remaining edges based on the SIR model.  The edges we remove at this stage are also marked with an $X$.  In the final row, \textbf{j)} represents the results of the epidemic without the intervention, while \textbf{k)} represents the results of the same epidemic with an intervention removing two specific edges.}
\end{figure}

%% FIGURE 2
\begin{figure}\label{fig:outline}
\includegraphics[width=\textwidth]{../figures/figure8} %Probably turn the coloration into a legend.
\caption{Diagram of the steps in the pruning process.  We start with the complete graph \textbf{a)}.  We prune according to the SIR model ignoring the intervention to obtain a potential network \textbf{b)}.  We then can choose to apply an intervention if we want \textbf{b${}^*$)}.  In either case, we then realize the network \textbf{c)} or \textbf{c${}^*$)}.}
\end{figure}

Single-world inference begins with a network in which there is a node for each person I at each time step.
In this example, there are three states that each node might occupy: susceptible, infected, or recovered; and two potential types of edges: transmissions or recoveries.
At each time step, each person (node) will have a recovery edge connecting their current self to their future self in the next time step.
Additionally at each time step, each person will have an edge connecting them to each other person in the network one time step ahead, representing an infectious contact.
This is the complete network.

From the the complete network, we begin percolating. 
%% THIS BELONGS SOMEWHERE NOW (Figure \ref{fig:pruning}).
To prune the network, we update the nodes based on what we know from all previous time steps.
Although each node can potentially be susceptible, infected, or recovered, however, it is often possible to eliminate one or more states.
%For the states we cannot rule out as impossible, we say they are possible as they may occur in the intervention case or the non-intervention case. %% THIS DOESNT ADD ANYTHING. NOT IMPOSSIBLE NODES ARE POSSIBLE... TAUTOLOGICAL? 
However, some of these edges may be mutually exclusive.
It is possible for a person to be susceptible (and get infected), or infected (and recover) and it might be true that we cannot rule out either possibility, but we know that it is impossible to do both in the same time step.
Our method addresses these mutually exclusive events later. %% WHERE LATER? YOU SHOULD SAY 
There are three ways to eliminate possible states of a node.
First, the state of a node at time $t=0$ cannot be anything other than the initial state of that node.
Next, anyone who cannot currently be infected, and does not have the potential to have an infectious contact between time steps, cannot be an infected node.
Finally, anyone who would have recovered if they were infected, but cannot have an infectious contact if the were not infected (i.e., susceptible) between time steps, cannot be an infected node.

With information about the current state of the nodes, we can see that certain types of edges are impossible (e.g. a person cannot recover if they could not have been infected).
We then remove those edges.
For transmissions, we can rule out any edge where either of the nodes at the previous time step do not have compatible states.
For recoveries, we can rule out any edge where the person cannot be currently infected.

Finally, for the edges that have not yet been ruled out, we refer to the stochastic SIR model to determine which edges to keep.
This includes some edges which will later be removed, as they are dependent on their conditions being satisfied (i.e., a condition of recovery is to be currently infected).
We keep transmission edges with probability $\beta$ and recovery edges with probability $\gamma$.
We call this the pruned network.

Now that we have the pruned network, applying our intervention is fairly straightforward.
We repeat the pruning procedure with one difference: now, instead of keeping track of potentialities, we update nodes with their actual state.
This new graph is similar to a transmission tree, as it contains all of the events as they happen in time.
In fact if we remove every recovery edge and any disconnected nodes, then this new graph is a transmission tree.
Therefore it can be used to determine the course of the epidemic.
It is important to note that the intervention is applied after the SIR model has pruned the network.
Indeed, as long as we have the pruned graph, it can be used multiple times for different interventions.

We evaluate the effects of four control measures on influenza cases averted: hand-washing, vaccination, antivirals, and a null intervention.
We selected these example interventions to highlight different ways in which interventions can interact with single-world inference.
We assume that for hand-washing, everyone washes their hands and that it reduces the transmissibility of influenza by $1\%$.
For vaccination intervention, we choose $25\%$ of people to be vaccinated at the beginning of the season and assume it fully protects $33\%$ of those vaccinated
For the antiviral intervention, we choose $25\%$ of people to take antivirals when they first become infected, assume that it reduces their average duration of infection by $0.7$ days \cite{oseltamivir:2014}, and model this reduction in duration of infection as an additional $36.1\%$ chance of recovering at each time step.
To highlight the differences between single-world and multiple-world inference, we include a null intervention in which we do not intervene.
While this is not an intervention, it effectively demonstrates the ability of our method to reduce the process model variance.
We run $1,000$ simulations of each type of inference, for each intervention, on a population of $4,000$ for $100$ days and calculate the estimated number of cases averted.

\subsection{General Framework}\label{subsection:general}

We extend the single-world inference example to a general class of compartmental models and generic interventions by outlining the differences between the general method and the influenza example.
Single-world inference allows a general class of compartmental models. 
We define these as a system of difference equations with $K$ compartments governed by \[\Delta x_{j} = \sum_{i \neq j} \alpha_{i,j}x_i + \sum_{i \neq j, k} \alpha_{i,j,k} x_ix_k.\]
Where $\alpha_{i,j}$ is the probability that a person transitions from state $j$ to state $i$, and $\alpha_{i,j,k}$ is the probability that a person transitions from state $j$ to state $i$ after contacting with an individual in state $k$.
Each transition is represented both as entering a compartment $\alpha > 0$, and leaving a compartment $\alpha < 0$.
Since we are only interested in people entering new states, we only track positive values of $\alpha$.
We can rephrase our influenza example in these new terms where,
\begin{align*}
    x_1 &= S \\
    x_2 &= I \\
    x_3 &= R
\end{align*}
and $\alpha_{2,1,2} = \beta $, and $\alpha_{3,2} = \gamma$ all other $\alpha_{i,j,k}$ and $\alpha_{i,j}$ are equal to $0$.

We construct the graph similarly to the influenza example.
Where the nodes now have $K$ possible states, $x_1 \dots x_K$.
For each $\alpha_{i,j,k}$ and $\alpha_{i,j}$, we have a different type of edge.
For $\alpha_{i,j,k}$ we connect each node to the node for each other person at the next time with an edge of that type.
We call these edges interaction edges.
For $\alpha_{i,j}$ we connect each node the node for the same person at the next time step with an edge of that type.
We call these edges transition edges.
Unlike in the influenza example, there may be multiple edges connecting the same pair of nodes.

We prune the network the same way: for each time step, we update the possible states of the nodes, prune edges we do not need to test, and testing the remaining edges according to the model.
While the steps are the same, some of the reasoning needs to be generalized.
When updating the state of the nodes, we relied on assumptions about the intervention to let us eliminate possible states.
We can generalize those assumptions to allow us to prune the same way.
We only allow interventions such that that the intervention does not add any edges to the graph.
This assumption allows us to say that if a person is not in a state, and no transition edge or interaction edge would cause them to move to that state, then they are not in that state.
We weaken this assumption by allowing an intervention to add transitions, but only to states which the person will not leave (e.g., Removed in the SIR model).
We require that our intervention does not remove any transition edges from the graph.
This assumption allows us to say that if a person may be in a state, and a transition edge would cause them to leave that state at the next time step, but no transition or interaction edges would cause them to enter that state at the next time step, then that person is not in that state. %%% HUH?? NO IDEA WHAT YOU'RE SAYING HERE
These assumptions are more restrictive in the general case than the SIR case, but still include most common interventions.

To prune the generic network, we can prune any edge associated with $\alpha_{i,j}$ if it is impossible for the source to be in state $i$ and any edge associated with $\alpha_{i,j,k}$ if it is impossible for the source to be in state $k$.
We also prune an edge associated with $\alpha_{i,j,k}$ if it is impossible for the node representing the same person as the destination at the previous time step as the destination to be in state $i$. %%% HUH?? NO IDEA WHAT YOU'RE SAYING HERE
For the edges that remain, we keep them with probability $\alpha_{i,j}$ or $\alpha_{i,j,k}$.
We now have our pruned graph.

We then do a second pruning to account for the intervention.
The second pruning is almost identical to the influenza case, but there is an additional problem we might encounter.
It is now possible that two different edges affect the same person at the same time, changing their state to two different states.
Worse, those edges might both have their conditions met.
As an example, in an age cohort model, a susceptible child might be infected, and become an adult the same time step .
Unfortunately, there are many ways to handle this type of conflict, and the right way depends on the model.
In the age case, the correct answer is that both events happen, the susceptible child becomes an infected adult.
In a different case, the answer may be to choose randomly between the two events, or that one of the events takes priority over the other (a dead adult is just dead for example).


% FIX ME tie up paragraph.-- It doesn't help to have notes like this if you don't do them. 
%We apply our intervention and extract the actual states in almost the same way as the influenza example, however, we need to account for an additional circumstance:
%It is now possible that two different events affect the same person at the same time, changing their state to two different states.
%As an example, in an age cohort model, someone might be infected, and move to a higher age state at the same time.
%There are many reasonable ways to handle this, and the right one depends on the model.
%In the age case, the correct answer is that both events happen.
%In a different case, the answer may be to choose randomly between the two events, or something even more complicated.
%%% ISN'T THIS THE EXACT SAME AS THE PARAGRAPH ABOVE? I THOUGHT YOU FIXED THIS SECTION BEFORE SENDING IT OUT? YOU SHOULD MAKE SURE TO READ EVERYTHING THROUGH BEFORE SENDING IT OUT. (I commented it out to see how much space you have left)

\section{Results}

%% FIGURE 3
\begin{figure}\label{fig:boxplots}
\centering
\includegraphics[width=.31\textwidth]{../figures/intervention-effects-raw-boxplots}
\includegraphics[width=.31\textwidth]{../figures/intervention-effects-combined-boxplots}
\caption{\textbf{Left)} Boxplots of simulations results under the each intervention scenario.  Notice that all of the boxplots overlap, which implies that from the population perspective, the effects of the interventions are all insignificant.  \textbf{Right)} The single-world estimate of cases averted for each intervention scenario.  Unlike the right hand figure, very few of the boxplots overlap with The null intervention.}
\end{figure}

%% FIGURE 4
\begin{figure}\label{fig:epicurve}
\centering
\includegraphics[width=.7\textwidth]{../figures/intervention-effects-time-series-susceptible-switched.pdf}
\caption{Time series showing the number of cases averted at each time caused by the intervention calculated using single-world inference and multiple-world inference.  We observe that single-world inference performs how you would expect the interventios to perform; cases prevented are almost always positive, and many of the cases prevented stay prevented (though not all). On the other hand, multiple-world inference generally has a very different shape, apparently possibly causing a large number of cases in the middle of the epidemic, and a lot of uncertainty surrounding the peak of the epidemic.}
\end{figure}

We define and demonstrate single-world inference and apply it to determining the impact of three interventions on the number of cases averted.
We compare the outcomes of single-world inference to the current standard way to simulate the effects of an intervention, multiple-world inference.
We find that single-world inference generally has the same point estimates as multiple-world inference, but with narrower confidence intervals.
We also see that multiple-world inference has some nonsensical results that single-world inference avoids.

For single-world inference, we found that every non-null intervention had a significant ($p<0.05$) number of cases averted: Null $0$ (CI $0-0$), Antivirals $819.125$ (CI $206 - 2899.05$), Social Distancing $153.594 $ (CI $12 - 298$), Vaccination $ 284.955 $ (CI $130 - 438$).
For multiple-world inference, we found that all but one non-null intervention had a significant ($p<0.05$) number of cases averted: Null  -$5 $ (CI  -$195 - 183$), Antivirals $814.185$ (CI $176 - 2913 $), Social Distancing $148.654$ (CI -$69 - 355$), Vaccination $280.015$ (CI $72 - 499$).
In addition to estimates of cases averted, we observe that multiple-world has an unintuitive effect on the shape of the epidemic (Figure \ref{epicurve}).

We released a software package, \texttt{counterfactual}, on \texttt{CRAN}, %NOTE: This is not true yet
which implements single-world inference as described in section \ref{sec:methods}.
It follows the assumptions and procedures outlined above, but does not enforce those assumptions, or deal with conflicting state changes.
The amount of space and time pruning takes in the software package is $\mathcal O(E)$, where $E[O(E)]$ which is on average $O(t(n^2\sum \alpha_{i,j,k} + n \sum\alpha_{i,j}))$ per simulation, or $O(t(n^2\beta + n\gamma))$ in the case of the SIR model.
The amount of time the interventions take is also $\mathcal O(E)$, but the constant is lower (Table \ref{table:performance}).
This shows us that the number of infectious compartments, the population, number of times, and number of simulations all impact the amount of time taken, but population has the biggest impact.

\section{Discussion}
In this paper, we present a method that can reduce uncertainty and variability in the underlying process model to reflect variability only in the intervention itself.
We demonstrate the ability of our method to factor away inter-epidemic variation via simulation and show that through precise simulation, our method avoids the issue of counterintuitive results. 
Single-world inference is modular and can be applied to a broad class of models and interventions, simultaneously; most of which are supported by our software package.

Inherently, our method sets out to address a different question than most simulation-based methods. 
Traditional methods simulate two completely independent epidemics in the same population, from the same parameters, with one epidemic intervened upon while the other is not. 
These address important policy questions: if you were to see another epidemic in the same population, what would the impact of an intervention look like on that epidemic. 
While the questions that traditional simulation-based method can address are vitally important, our method sets out to address a more precise question. 
Given two precisely identical epidemics what is the effect of the intervention. 
This method precisely isolates the expected intervention impact which allows for a better comparison of interventions or a better understanding of the true differences between control programs.

Additionally, our model does have some limitations.
Some reasonable interventions are out of reach as they do not follow the assumptions we make about interventions as part of pruning.
For a particular intervention, we can make stronger assumptions, even going so far as to run the intervention as we prune, keeping only the actual state as possible.
We think there may be a better middle ground, one that allows more interventions, and also improves performance.
It would be nice to be able to make probabilistic assumptions about the interventions, which would guarantee success with high probability.
Even with these changes, interventions that fundamentally change the way the disease behaves (changes the underlying model) are beyond the scope of this method.

Another set of limitations is that discrete time compartmental models are not cutting edge modeling techniques.
Our method is discrete time, but Gillespie algorithms would allow it to be extended into continuous time.
Our method considers compartmental models, but it could be extended to network or agent based models.
Our method also uses a fixed difference equation, a stronger limitation than we would like.
We can construct interventions to overcome this limitation to some extent, but it could be improved.
Moreover, to account for all of them would result in  combinatorial explosion and quickly become computationally intractable. % FIX ME: is this sentence actually true?
%%% FOLLOW THE COMMENTS I SENT EARLIER TO COMBINE THESE TWO PARAGRAPHS INTO ONE

Although we present a methodology for removing process error by simulating an epidemic with and without interventions, our model is too computationally intensive to include secular changes that are not captured directly by an intervention such as the change in behavior of individuals in response to disease spread (i.e., cite Ebola). 
These secular changes represent a combinatorial explosion of potentialities that make this method computationally impossible. 
For example, the influenza SIR model presented here \textbf{[josh confirm this]} 18 petabytes. 
Going forward, an important extension of the work at the intersection of of computer science and epidemiology is to improve upon this method such that it remains logically consistent with fewer storage requirements. 
Interesting work remains to be done to improve our ability to make inference without continuing to significantly increase the computational power needed.
We have developed a method for utilizing computational power to address more a precise counterfactual question about interventions. 
Single-world inference allows us to more precisely understand the effects of interventions within the context of a single outbreak. 
With precise understanding, we can better determine when an intervention is effective, and when it is spurious. Single-world inference also allows us to better compare interventions, and aids in choosing the best intervention from many possibilities. 
Single-world inference also mark a new way of leveraging computational resources to solve epidemiological problems.

\enlargethispage{20pt}

\ethics{This research did not require ethics approval as it used simulated  data.}

\dataccess{Code used for the included analyses are available upon publication.}

\aucontribute{Conception and design of study: JL\\
Development and/or verification of analytic methods: JK and JL\\
Analysis and/or interpretation of results: JK, LTK, and JL\\
Drafting the manuscript: JK and LTK\\
Revising the manuscript: JK, LTK, CJM, and JL \\
Approval of the final manuscript: JK, LTK, CJM, and JL}

\competing{We have no competing interests.}

\funding{Insert funding text here.}

\ack{Insert acknowledgment text here.}



\bibliographystyle{apalike}
\bibliography{bibliography}
\end{document}
