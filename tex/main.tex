
%% Author_tex.tex
%% V1.0
%% 2012/13/12
%% developed by Techset
%%
%% This file describes the coding for rsproca.cls
\documentclass[PTRSB]{rsos}
\makeatletter\if@twocolumn\PassOptionsToPackage{switch}{lineno}\else\fi\makeatother

%Publisher: Royal Society
%Template Provided By: Typeset

\usepackage{amsmath,tabulary,graphicx,multicol}
\usepackage[utf8]{inputenc}
\long\def\ack#1{{\vskip5.5pt\noindent \fontsize{8}{11}\selectfont#1}}
\renewcommand{\neg}{{}^-}

%%%%%%%%%%%%%%%%%%%%%%%%%%%%%%%%%%%%%%%%%%%%%%%%%%%%%%%%%%%%%%%%%%%%%%%%%%
% Following additional macros are required to function some 
% functions which are not available in the class used.
%%%%%%%%%%%%%%%%%%%%%%%%%%%%%%%%%%%%%%%%%%%%%%%%%%%%%%%%%%%%%%%%%%%%%%%%%%
\usepackage{url,multirow,morefloats,floatflt,cancel,tfrupee}
\makeatletter


\AtBeginDocument{\@ifpackageloaded{textcomp}{}{\usepackage{textcomp}}}
\makeatother
\usepackage{colortbl}
\usepackage{xcolor}
\usepackage{pifont}
\usepackage[nointegrals]{wasysym}
\urlstyle{rm}
\makeatletter

%%%For Table column width calculation.
\def\mcWidth#1{\csname TY@F#1\endcsname+\tabcolsep}

%%Hacking center and right align for table
\def\cAlignHack{\rightskip\@flushglue\leftskip\@flushglue\parindent\z@\parfillskip\z@skip}
\def\rAlignHack{\rightskip\z@skip\leftskip\@flushglue \parindent\z@\parfillskip\z@skip}


\if@twocolumn\usepackage{dblfloatfix}\fi
\usepackage{ifxetex}
\ifxetex\else\if@twocolumn\usepackage{dblfloatfix}\fi\fi

\AtBeginDocument{
\expandafter\ifx\csname eqalign\endcsname\relax
\def\eqalign#1{\null\vcenter{\def\\{\cr}\openup\jot\m@th
  \ialign{\strut$\displaystyle{##}$\hfil&$\displaystyle{{}##}$\hfil
      \crcr#1\crcr}}\,}
\fi
}

%For fixing hardfail when unicode letters appear inside table with endfloat
\AtBeginDocument{%
  \@ifpackageloaded{endfloat}%
   {\renewcommand\efloat@iwrite[1]{\immediate\expandafter\protected@write\csname efloat@post#1\endcsname{}}}{}%
}%

\def\BreakURLText#1{\@tfor\brk@tempa:=#1\do{\brk@tempa\hskip0pt}}
\let\lt=<
\let\gt=>
\def\processVert{\ifmmode|\else\textbar\fi}
\let\processvert\processVert

\@ifundefined{subparagraph}{
\def\subparagraph{\@startsection{paragraph}{5}{2\parindent}{0ex plus 0.1ex minus 0.1ex}%
{0ex}{\normalfont\small\itshape}}%
}{}

%%% These are now gobbled, so won't appear in the PDF.
\newcommand\role[1]{\unskip}
\newcommand\aucollab[1]{\unskip}
  
\@ifundefined{tsGraphicsScaleX}{\gdef\tsGraphicsScaleX{1}}{}
\@ifundefined{tsGraphicsScaleY}{\gdef\tsGraphicsScaleY{.9}}{}
%%% To automatically resize figures to fit inside the text area
\def\checkGraphicsWidth{\ifdim\Gin@nat@width>\linewidth
	\tsGraphicsScaleX\linewidth\else\Gin@nat@width\fi}

\def\checkGraphicsHeight{\ifdim\Gin@nat@height>.9\textheight
	\tsGraphicsScaleY\textheight\else\Gin@nat@height\fi}

\def\fixFloatSize#1{}%\@ifundefined{processdelayedfloats}{\setbox0=\hbox{\includegraphics{#1}}\ifnum\wd0<\columnwidth\relax\renewenvironment{figure*}{\begin{figure}}{\end{figure}}\fi}{}}
\let\ts@includegraphics\includegraphics

\def\inlinegraphic[#1]#2{{\edef\@tempa{#1}\edef\baseline@shift{\ifx\@tempa\@empty0\else#1\fi}\edef\tempZ{\the\numexpr(\numexpr(\baseline@shift*\f@size/100))}\protect\raisebox{\tempZ pt}{\ts@includegraphics{#2}}}}

\renewcommand{\includegraphics}[1]{\ts@includegraphics[width=\checkGraphicsWidth]{#1}}
\AtBeginDocument{\def\includegraphics{\@ifnextchar[{\ts@includegraphics}{\ts@includegraphics[width=\checkGraphicsWidth,height=\checkGraphicsHeight,keepaspectratio]}}}

\def\URL#1#2{\@ifundefined{href}{#2}{\href{#1}{#2}}}

%%For url break
\def\UrlOrds{\do\*\do\-\do\~\do\'\do\"\do\-}%
\g@addto@macro{\UrlBreaks}{\UrlOrds}

\@ifundefined{quoteAttrib}
	{\long\def\quoteAttrib#1{\par\raggedleft\itshape#1\unskip}}
	{}

\@ifundefined{titlequoteAttrib}
	{\long\def\titlequoteAttrib#1{\list{}{\topsep-3pt\leftmargin.5in\rightmargin0pt}%
  \item\relax---\upshape#1\endlist}}{}

\renewenvironment{quote}
	{\list{}{\leftmargin.5in\rightmargin\leftmargin}%
  \item\relax}
  {\endlist}

\newenvironment{title-quote}
	{\list{}{\fontsize{10pt}{12pt}\selectfont\leftmargin.5in\itshape\rightmargin\leftmargin}%
  \item\relax}
  {\endlist}


\makeatother
\def\floatpagefraction{0.8} 
\def\dblfloatpagefraction{0.8}
\def\style#1#2{#2}
\def\xxxguillemotleft{\fontencoding{T1}\selectfont\guillemotleft}
\def\xxxguillemotright{\fontencoding{T1}\selectfont\guillemotright}
%%%%%%%%%%%%%%%%%%%%%%%%%%%%%%%%%%%%%%%%%%%%%%%%%%%%%%%%%%%%%%%%%%%%%%%%%%



\Journal{Phil. Trans. R. Soc. B }

\usepackage[nolists,nomarkers]{endfloat}

%%%% Article title to be placed here
\title{Single-World Counterfactual Inference}

\author{%%%% Author details
Joshua Kaminsky$^{1}$, Lindsay T. Keegan$^{1}$,\\C. Jessica E. Metcalf$^{12}$ and Justin Lessler$^{1}$}

%%%%%%%%% Insert author address here
\address{$^{1}$Department of Epidemiology, Johns Hopkins Bloomberg School of Public Health, Baltimore, MD, USA\\
$^{2}$Department of Ecology and Evolutionary Biology, Princeton University, Princeton, NJ, USA}

%%%% Subject entries to be placed here %%%%
\subject{Epidemiology, Infectious Disease Modeling}

%%%% Keyword entries to be placed here %%%%
\keywords{Counterfactuals, Infectious disease modeling, Infectious disease Dynamics, Network Modeling}

%%%% Insert corresponding author and its email address}
\corres{Justin Lessler\\
\email{justin@jhu.edu}}

\begin{document}
%%%% Abstract text to be placed here %%%%%%%%%%%%
% Recount words
\begin{abstract} % 154 words MAX 183 words
  Determining the effects of an intervention on a disease outbreak is an important problem in epidemiology. %15 words
  However, multiple confounders and sources of noise make it difficult to address. %12 words
  Here we present a method that addresses an important source of noise, stochastic variation within the same population. %18 words
  This allows for improved ability to select the best interventions for an outbreak %18 words
  Our method, like many others, employs compartmental models and extends an individual-level effect of the intervention to the population-level effect. %22 words
  We eliminate stochastic variation between epidemics by tracking both what happens with intervention and what happens without intervention within a single simulated epidemic. %23 words
  To simultaneously keep track of both states, we store all transition states that are possible either with or without the intervention on a network. %24 words
  We can actualize this network to recover the events that occur both with and without the intervention. %17 words
  We demonstrate our method on an influenza epidemic and compare the results of our method to an existing method. %19 words
  We compare across three interventions: hand-washing, vaccination, antivirals, and a control; and evaluate performance on the estimate of cases averted. %21 words
  We find that our method reduces the variance of all estimates, and in one case makes the results statistically significant. %20 words
  Although our method is more computationally intensive than traditional methods, it is still computationally tractable for many common use cases, making it a useful tool for selecting interventions. %28 words
\end{abstract}
%%%%%%%%%%%%%%%%%%%%%%%%%%%


%%%%%%%%%%%%%%% End of first page %%%%%%%%%%%%%%%%%%%%%

\maketitle

\begin{multicols}{2}
\section{Introduction}
Dynamic models are frequently used to assess the likely impact of disease control strategies.
These exercises range from modeling the impact of a new intervention or strategy on an established pathogen [CITE], to models of the containment and control of emergent epidemics \cite{lessler-et-al:2016}.
While deterministic models are frequently used \cite{fraser-et-al:2004}, stochastic simulations are increasingly common as they can account for both uncertainty in the underlying parameters and the random nature of the disease process \cite{ferguson-et-al:2003}.
In both stochastic and deterministic models, the impact of interventions are typically determined by comparison of simulations with and without the intervention.

In the deterministic setting, this comparison is straight forward, as with a given set of parameters and starting conditions the epidemic will always behave exactly the same; hence any comparison between an intervention scenario and its non-intervention ‘counterfactual’ can only be attributed to the intervention itself.
When stochastic models are used, things become more complicated.
Typically, two sets of simulations are conducted, one with the intervention and one without, and then the distribution of outcomes from the two sets are compared to estimate the intervention effect.
Because these are independent sets of simulations, there may be some simulations in the non-intervention scenario where the disease dies off quickly due to random chance, and fewer cases occur than the majority of intervention simulations.
Likewise, there may be cases in the intervention scenario where large numbers are infected through sheer ‘bad luck’ for the virtual populations involved.
If these stochastic effects are large, they may lead uncertainty intervals in effect estimates to include the intervention having no effect, or even a negative impact, even if the intervention is modeled in such a way that it can only have a beneficial effect.
For example, the results of a study of measles vaccination strategies by Lessler et al. appears to leave open the possibility that more cases of measles could occur in a country if supplementary vaccination activities were conducted than if those campaigns had not occurred.
Likewise, Rivers et al. appears to leave open the possibility that more cases of Ebola could occur in a country if pharmaceuticals were introduced to help patient survivability than if no pharmaeuticals were introduced.
% Potentially find one more example.
These effects will be exacerbated if the processes being modeled are complex or we are simultaneously sampling over parameter uncertainty.

The results of this approach have a very specific interpretation: they represent the difference between what we expect to be observed in an uncontrolled epidemic compared to a completely independent epidemic where the intervention occurred (conditional on the starting conditions).
However, what we often want to know is what would have happen had the intervention occurred in the exact same epidemic.
This is equivalent to the problem of counterfactual inference in randomized trials and observational studies, where we take one set of individuals (or populations) as a stand in for what would have happened in another in the counterfactual situation that they had experienced some exposure.
A number of techniques of trial design and statistical analysis have been developed to help such real world studies better approximate the true counterfactual situation (\cite{hudgens-halloran:2008,murray-et-al:2017,buchanan:2014} are just a few examples of the large literature on the subject). %Potentially find another reference than the dissertation
However, in computational simulations it is possible to take a more exact approach.

Here, we present a method for simulation of direct counterfactuals to stochastic simulations using principles borrowed from the percolation approach to epidemic simulation \cite{miller-book}.
We illustrate this ‘single-world’ approach using simulations of interventions against an influenza like illness, and outline how the general approach can be adapted to a wide variety of disease systems and simulation methodologies.

\section{Methods}
\subsection*{Overall Approach}
%% Key points
Our proposed method separates modeling the epidemic and the intervention so we can apply them sequentially ensuring the intervention and non-intervention results come from the same epidemic.
In order to separate the epidemic from the intervention, we need to model events that occur in the intervention case, but not the non-intervention case, and vice versa.
Because some of the events we model will not happen in at least one case, any appropriate modeling approach must consider potential events rather than actual events.

%% Graph
Our approach begins with an implicit set of all possible events, which we think of as a large graph where each edge is an event, and each node is a person at a time. %The model implies this
We call this graph the complete graph.
The underlying epidemic model assigns each event (edge) a probability of occuring conditional on the states of people at the previous time for any potential intervention.
We could stochastically determine whether each event occurs under those same conditions with and without the intervention, removing those events which do not occur.
We call the remaining graph the epidemic potential graph.
We can then apply initial conditions to determine which events actually occur, and the state of each person at each time.
We call this graph the realized epidemic graph.

%% Pruning
If the above approach were computationally feasible, we would apply it and be finished.
However the method is intractable except in small populations for simple models (more detail in results).
By making assumptions about the way interventions work, we are able to prune enough potential events for the problem to become tractable.

\subsection*{General Framework}
We define a general compartmental model as a system of difference equations with $K$ compartments governed by 
\begin{align*}
\Delta x_{j} &= \sum_{i \neq j} \alpha_{i,j}x_i + \sum_{i \neq j, k} \alpha_{i,j,k} x_ix_k.
\end{align*}
We can interpret $\alpha_{i,j} > 0$ as the probability a person transitions to state $j$ conditional on being in state $i$, and $\alpha_{i,j,k} > 0$ as the probability that a person transitions to state $j$ conditional on being in state $i$ another specific person being in state $k$.
Then, these $\alpha_{i,j}$ and $\alpha_{i,j,k}$ each represent a type of event corresponding to several edges in the graph: $\alpha_{i,j,k}$ edges connect a person at one time to a different person at the next time, and $\alpha_{i,j}$ edges connect a person at one time to the same person at the next time.
Each person also has the potential to be in each of the $K$ compartments at each time, and we use the nodes of our graph to keep track of this.
%%%% Explanation of how we make edges out of them probably seems right about now.

%% Pruning Overview
To construct the epidemic potential graph, we prune the complete graph by applying three steps at each time: we prune events that deterministically cannot occur; we stochastically prune the remaining events according to the model; and we eliminate potential states from the nodes according to our assumptions (Figure \ref{fig:pruning}).
If we make no assumptions about pruning, then this collapses back to the method presented in the overview and any potential intervention is allowed.
If we assume that each node only has one potential state, its actual state, then this method collapses back to the compartmental model presented above and no intervention is allowed.
We want assumptions general enough to admit interventions of interest, but not so general the problem becomes intractable.

%% FIGURE 1
\begin{figure}[hp]
\includegraphics[width=\textwidth]{../figures/figure6.pdf}
\caption{\textit{‘Single-world’ simulation process.}
  To simulate an initial epidemic and interventions, we start with a (implicit) graph where each node represents the possible states a person at particular time in the epidemic and edges represent possible events that can change a nodes state (a).
  In the above example, a person can be susceptible (purple pie slice), infectious (orange) or immune (green), and the possible events are infection (orange arrows) and recovery (green arrows).
  We start by assigning each individual in the population a state at time 0 (b), here infecting person 1.
  We next remove all edges that are deterministically impossible based on the initial state (c).
  We then prune those edges stochastically selected not to occur by our underlying infection model (d).
  The possible states of each person for time 1 based on the edges still existing in the graph, so each person’s set of potential states encompasses those that could exist if each of the previous set of events did or did not exist (e).
  We now repeat steps c-e for events between time 1 and time 2 (f-h), and times 2 and 3 (i-k) and so on.
  Note that when we prune the edges between steps 1 and 2 (i) we still keep the those for persons 1 and 2 infecting person 3, even though person 3 has already been infected, because susceptibility is still a possible state for person 3 at time 2 in an intervention scenario (i.e., elimination of event edges could lead to a scenario where person 3 is susceptible at time 2).
  This final [potential epidemic] graph can then be used to obtain simulated epidemics with and without interventions (see Figure \ref{fig:actualizing}).
}
\label{fig:pruning}
\end{figure}

%% Pruning assumptions
%% This paragraph is not great, but i'm not sure how to fix it currently.
%% The goal is to introduce the strong pruning assumptions: do not alter the initial conditions, do not remove $\alpha_{i,j}$ edges, and do not add edges.
%% These assumptions are not our actual assumptions, but we pretend they are.
%% Our actual assumptions are that the pruned graph for the intervention will have all the same edges as the graph that we get with this assumption ^^.
%% This allows us to add terminal transitions, and transition into a state they would already have been able to be in. (We use the first of these but not the second).
When pruning, we assume that interventions do not add any events to the graph, do not alter the initial conditions, and do not remove events associated with $\alpha_{i,j}$ from the graph.
As stated, these assumptions are too strong, and prevent many standard interventions (vaccination, for example, adds transitions to a vaccinated compartment).
We \emph{only} use these assumptions to say events deterministically cannot happen, and so we only require that 
Adding a transition to a state that a person never leaves, or a state that was otherwise among their potential states does not cause any new events to be deterministically impossible.

To construct the realized epidemic graph from the epidemic potential graph, we prune the epidemic potential graph by applying three steps at each time: we prune events that did not occur, we stochastically determine if the intervention removes any events; and we determine the actual state of each person (Figure \ref{fig:acutalizing}).
The intervention plays two roles here, both preventing some events from happening in step $2$, and altering the states of individuals directly in step $3$.
Altering the states of individuals is equivalent to adding new events to the graph, and therefore restricted by our assumptions above.

%% FIGURE 2
\begin{figure}[hp]
\includegraphics[width=\textwidth]{../figures/figure10.pdf}
\caption{\textit{‘Single-world’ simulation process (continued).}
  To obtain a simulated epidemic including a potential intervention, we start with a potential epidemic graph (a) (see Figure \ref{fig:pruning} for how to create a epidemic potential graph).
  In the above example, a person can be susceptible (purple pie slice), infectious (orange) or immune (green), and the possible events are infection (orange arrows) and recovery (green arrows). %Potentially remove this
  There is only one possible state at time $0$, so we set the true state of each node to that possible state (b).
  We next remove all edges that are impossible based on the actual state (c). %This will never actually do anything, so could remove
  We then prune those edges stochastically selected not to occur by our underlying intervention model (d).
  The actual state of each person for time $1$ is the the color of any incoming edges, or its previous state if there are no incoming edges.
  We also allow the intervention to change the actual state of any node at this stage (e).
  We now repeat steps c-e for events between time 1 and time 2 (f-h), and times 2 and 3 (i-k) and so on.
  This final [epidemic] graph can be used to extract our outcome of interest.
  Any epidemic graphs made from the same epidemic potential graph represent the results of different interventions in a `single-world'.
}
\label{fig:actualizing}
\end{figure}

At this stage, we also need to account for the possibility of two conflicting events occuring (e.g. a person may become infected and grow older in an age cohort model).
In some cases the events interact (the person grows older and is infected), in other cases one event takes precedence over the other (a person who dies and gets infected is dead).
It is up to the modeler to account for these cases, and determine what should happen.

%% Missing from this section: Realized epidemic graph includes a transmission tree as a subgraph, and contains the case counts
%% This sentence below should go somewhere
%We can obtain many quantities of interest from the realized epidemic graph, including transmission trees, compartment populations over time, and anything else you can get from an individual SIR level model.

\subsection*{Influenza Example}
To demonstrate single-world inference, we construct an influenza example using a standard Susceptible (S), Infected (I), Immune (R) compartmental model.
By modeling influenza transmission within a single season we do not account for loss of immunity and we assume that births and deaths are negligible.
We allow the probability of an infectious contact between any particular susceptible-infected pair to be $\beta = 0.78$ per day and the probability of an infected individual recovering at any time step to be $\gamma = 0.44$ per day \cite{forsberg-white-et-al:2009}.

We evaluate the effects of four control measures on influenza cases averted: hand-washing, vaccination, antivirals, and a null intervention.
We selected these example interventions to highlight different ways in which interventions can interact with single-world inference.
We assume that hand-washing reduces transmissibility by $1\%$, and therefore removes $1\%$ of transmissions.
For vaccination, we chose $8.25\%$ of people to be fully vaccinated against influenza, and therefore begin as Vaccinated instead of Susceptible.
For the antiviral intervention, we choose $25\%$ of people to take antivirals when they first become infected, assume that it reduces their average duration of infection by $0.7$ days \cite{oseltamivir:2014}.
We model this by choosing $25\%$ of people at the start of the epidemic, and if they become infected, they have a $36.1\%$ additional chance to recover at each time step.
To highlight the differences between single-world and multiple-world inference, we include a null intervention in which we do not intervene.
While this is not an intervention, it effectively demonstrates the ability of our method to reduce the process model variance.
We run $1,000$ simulations of each type of inference, for each intervention, on a population of $4,000$ for $100$ days and calculate the estimated number of cases averted.

\section{Results}

%% TABLE 1
\begin{table}
\caption{Space and time complexity results for a brute-force single-world inference, our single-world inference (first and second prunings), and multiple-world inference.}
\begin{tabular}{|l|l|l|}
  \hline
  Task & Space Use & Time / simulation\\\hline
  Single-world inference without pruning & $58.4$ TB & \textemdash \\\hline
  Construct Potential Network & $1.73$ GB & 54 seconds \\\hline
  Actualize Potential Network & $1.3-1.4$ KB & $16-20$ seconds \\\hline
  Multiple-World Inference& $2.6-2.8$ KB &  22-24 seconds\\\hline
\end{tabular}
\label{table:performance}
\end{table}

%% FIGURE 3
\begin{figure}
\centering
\includegraphics[width=.31\textwidth]{../figures/intervention-effects-raw-boxplots-cropped}
\includegraphics[width=.31\textwidth]{../figures/intervention-effects-combined-boxplots-cropped}
\caption{\textbf{Left)} \textbf{Boxplots of simulations results under the each intervention scenario.}  Notice that all of the boxplots overlap, which implies that from the population perspective, the effects of the interventions are all insignificant.  \textbf{Right)} The single-world estimate of cases averted for each intervention scenario.  Unlike the right hand figure, very few of the boxplots overlap with The null intervention.}
\label{fig:boxplots}
\end{figure}

%% FIGURE 4
\begin{figure}
\centering
\includegraphics[width=\textwidth]{../figures/intervention-effects-time-series-susceptible-switched-cropped.pdf}
\caption{Time series showing the number of cases averted at each time caused by the intervention calculated using single-world inference and multiple-world inference.  We observe that single-world inference performs how you would expect the interventios to perform; cases prevented are almost always positive, and many of the cases prevented stay prevented (though not all). On the other hand, multiple-world inference generally has a very different shape, apparently possibly causing a large number of cases in the middle of the epidemic, and a lot of uncertainty surrounding the peak of the epidemic.}
\label{fig:epicurve}
\end{figure}

We define and demonstrate single-world inference and apply it to determining the impact of three interventions on the number of cases averted.
We compare the outcomes of single-world inference to multiple-world inference.
We find that single-world inference generally has the same point estimates as multiple-world inference, but with narrower confidence intervals.
We also see that multiple-world inference has some nonsensical results that single-world inference avoids.

For single-world inference, we found that every non-null intervention had a significant ($p<0.05$) number of cases averted: Null $ 0 $ (CI $ 0 $ \textemdash $ 0 $), Antivirals $ 632.133 $ (CI $ 318 $ \textemdash $ 2806 $), Social Distancing $ 151.961 $ (CI $ 9 $ \textemdash $ 295 $), Vaccination $ 284.449 $ (CI $ 149 $ \textemdash $ 431 $).
For multiple-world inference, we found that every all but one non-null intervention had a significant ($p<0.05$) number of cases averted: Null $ \neg4.94 $ (CI $ \neg196 $ \textemdash $ 184 $), Antivirals $ 627.193 $ (CI $ 292 $ \textemdash $ 2808 $), Social Distancing $ 147.021 $ (CI $ \neg67 $ \textemdash $ 354 $), Vaccination $ 279.509 $ (CI $ 82 $ \textemdash $ 495 $).
In addition to estimates of cases averted, we observe that multiple-world has an unintuitive effect on the shape of the epidemic (Figure \ref{fig:epicurve}).

We released a software package, \texttt{counterfactual}, on \texttt{CRAN}, %NOTE: This is not true yet
which implements single-world inference as described the methods section.
It follows the assumptions and procedures outlined above, but does not enforce our intervention assumptions or deal with conflicting state changes.
The amount of space and time pruning takes in the software package is $\mathcal O(E)$, where $E[O(E)] = O(t(n^2\sum \alpha_{i,j,k} + n \sum\alpha_{i,j}))$ per simulation, or $O(t(n^2\beta + n\gamma))$ in the case of the SIR model.
The amount of time the interventions take is also $\mathcal O(E)$, but the constant is lower (Table \ref{table:performance}).
This shows us that the number of infectious compartments, the population, number of times, and number of simulations all impact the amount of time taken, but population has the biggest impact.

\section{Discussion}
In this paper, we present a method that can reduce uncertainty and variability in the underlying process model to reflect variability only in the intervention itself.
We demonstrate the ability of our method to factor away inter-epidemic variation via simulation and show that through precise simulation, our method avoids the issue of counterintuitive results. 
Additionally, our method is tractable when applied to a broad class of models and interventions, simultaneously: most of which are supported by our software package.

Inherently, our method sets out to address a different question than most simulation-based methods. 
Traditional methods simulate two completely independent epidemics in the same population, from the same parameters, with one epidemic intervened upon while the other is not. 
These address important policy questions: if you were to see another epidemic in the same population, what would the impact of an intervention look like on that epidemic. 
While the questions that traditional simulation-based method can address are vitally important, our method sets out to address a more precise question. 
Given two precisely identical epidemics what is the effect of the intervention. 
This method precisely isolates the expected intervention impact which allows for a better comparison of interventions or a better understanding of the true differences between control programs.

Even at asking this different question, single-world inference has some inherent limitations.
Discrete time compartmental models are not cutting edge modeling techniques.
Our method is discrete time, but Gillespie algorithms would allow it to be extended into continuous time.
Our method considers compartmental models, but it could be extended to network or agent based models.
Extending the method in the above ways is non-trivial, even if we know the general way to proceed.

On the other hand, there are limitations caused by our assumptions about interventions that provide no clear path forward
Some reasonable interventions do not follow the assumptions about interventions outlined above in the general method.
Even if the intervention does follow our assumptions, the potential network may still be computationally intractable to store for complex models, large populations, or long times.
While we can address this concern for a particular intervention, the stronger assumptions we make, the less interventions follow them.
Worse, even if we ignore tractability some interventions are beyond the scope of this method.
We need to express the intervention in a way that the potential network captures, so if an intervention is too radical of a change, there is nothing we can do.

Going forward, an important extension of the work at the intersection of of computer science and epidemiology is to improve upon this method such that it remains logically consistent with fewer storage requirements. 
Interesting work remains to be done to improve our ability to make inference without continuing to significantly increase the computational power needed.
We have developed a method for utilizing computational power to address more a precise counterfactual question about interventions. 
Single-world inference allows us to more precisely understand the effects of interventions within the context of a single outbreak. 
With precise understanding, we can better determine when an intervention is effective, and when it is spurious.
Single-world inference also allows us to better compare interventions, and aids in choosing the best intervention from many possibilities. 
Single-world inference also mark a new way of leveraging computational resources to solve epidemiological problems.

\enlargethispage{20pt}

\ethics{This research did not require ethics approval as it used simulated  data.}

\dataccess{Code used for the included analyses are available upon publication.}

\aucontribute{Conception and design of study: JL\\
Development and/or verification of analytic methods: JK and JL\\
Analysis and/or interpretation of results: JK, LTK, and JL\\
Drafting the manuscript: JK and LTK\\
Revising the manuscript: JK, LTK, CJM, and JL \\
Approval of the final manuscript: JK, LTK, CJM, and JL}

\competing{We have no competing interests.}

\funding{Insert funding text here.}

\ack{Insert acknowledgment text here.}
\end{multicols}

\bibliographystyle{RS}
\begin{multicols}{3}
\bibliography{bibliography}
\end{multicols}
\end{document}
