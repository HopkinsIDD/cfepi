% tikzpic.teP
\documentclass[crop,tikz]{standalone}% 'crop' is the default for v1.0, before it was 'preview'

%Packages
\usepackage{xcolor}

%tikz libraries}}
\usetikzlibrary{shapes.geometric,positioning}

%Colors
\definecolor{susceptible}{RGB}{141,160,203}
\definecolor{recovery}{RGB}{102,194,165}
\definecolor{infection}{RGB}{252,141,98}
% \definecolor{intervention}{RGB}{231,138,195}
\definecolor{intervention}{RGB}{50, 102, 86}
\definecolor{nothing}{RGB}{225,225,225}

%New commands
\newcommand{\Leftscissor}{\large  X}
\newcommand{\Largescissor}{\Huge  X}

% Formatting macros:
\tikzstyle{node}=[minimum size= .5in,circle, draw]
\tikzset{intervened node fill/.style  args={#1,#2}{node,potential,circle split part fill={#1,#2}}}

\tikzset{potential/.style={opacity=.7}}
\tikzset{actual/.style={opacity=1}}
\tikzset{impossible/.style={opacity=.2}}

\tikzstyle{edge}=[diamond, draw]
\tikzstyle{intervention}=[fill = intervention]
\tikzstyle{contact}=[fill = infection]
\tikzstyle{recovery}=[fill = recovery]

\makeatletter
\tikzset{circle split part fill/.style  args={#1,#2}{%
    alias=tmp@name, % Jake's idea !!
    postaction={%
      insert path={
        \pgfextra{%
          \pgfpointdiff{\pgfpointanchor{\pgf@node@name}{center}}%
          {\pgfpointanchor{\pgf@node@name}{north}}%
          \pgfmathsetmacro\scale{1}
          \pgfmathsetmacro\insiderad{\pgf@y*\scale}
          \fill[#1] (\pgf@node@name.base) ([yshift=-\pgflinewidth]\pgf@node@name.north) arc
          (90:270:\insiderad-\pgflinewidth)--cycle;
          \fill[#2] (\pgf@node@name.base) ([yshift=\pgflinewidth]\pgf@node@name.south)  arc
          (270:450:\insiderad-\pgflinewidth)--cycle;            %  \end{scope}
        }}}}}
 \makeatother  
 
 \makeatletter
\tikzset{circle tri split part fill/.style  args={#1,#2,#3}{%
    alias=tmp@name, % Jake's idea !!
    postaction={%
      insert path={
        \pgfextra{%
          \pgfpointdiff{\pgfpointanchor{\pgf@node@name}{center}}%
          {\pgfpointanchor{\pgf@node@name}{north}}%
          \pgfmathsetmacro\scale{1}
          \pgfmathsetmacro\insiderad{\pgf@y*\scale}
          \fill[#1] (\pgf@node@name.base) ([yshift=-\pgflinewidth]\pgf@node@name.north) arc
          (90:210:\insiderad-\pgflinewidth)--(\pgf@node@name.center)--cycle;
          \fill[#2] (\pgf@node@name.base) ([yshift=-\pgflinewidth]\pgf@node@name.north)  arc
          (90:-30:\insiderad-\pgflinewidth)--(\pgf@node@name.center)--cycle; 
          \fill[#3] (\pgf@node@name.base) ([yshift=\pgflinewidth]\pgf@node@name.south)  arc
          (-90:-30:\insiderad-\pgflinewidth)--(\pgf@node@name.center)--cycle; 
          \fill[#3] (\pgf@node@name.base) ([yshift=\pgflinewidth]\pgf@node@name.south)  arc
          (270:210:\insiderad-\pgflinewidth)--(\pgf@node@name.center)--cycle;            %  \end{scope}
        }}}}}
 \makeatother

\begin{document}
\begin{tikzpicture}
\draw ( 0,-0 ) node (complete) {\includegraphics{../figures/figure5_pg_1.pdf}};
\draw (20,-0 ) node (initial)  {\includegraphics{figure5_pg_2.pdf}};
\draw ( 0,-15) node (t1s1)     {\includegraphics{figure5_pg_3.pdf}};
\draw (20,-15) node (t1s2)     {\includegraphics{figure5_pg_4.pdf}};
\draw (40,-15) node (t1s3)     {\includegraphics{figure5_pg_5.pdf}};
\draw ( 0,-27) node (t2s1)     {\includegraphics{figure5_pg_6.pdf}};
\draw (20,-27) node (t2s2)     {\includegraphics{figure5_pg_7.pdf}};
\draw (40,-27) node (t2s3)     {\includegraphics{figure5_pg_8.pdf}};
\draw ( 0,-39) node (t3s1)     {\includegraphics{figure5_pg_9.pdf}};
\draw (20,-39) node (t3s2)     {\includegraphics{figure5_pg_10.pdf}};
\draw (40,-39) node (t3s3)     {\includegraphics{figure5_pg_11.pdf}};

\node[below=of t1s3] (t1t21) {};
\node[below=of t1s1] (t1t22) {};
\node[below=of t2s3] (t2t31) {};
\node[below=of t2s1] (t2t32) {};

\draw[-latex,line width=2mm, color=black!40!white] (complete) -- (initial);
\draw[-latex,line width=2mm, color=black!40!white] (t1s1) -- (t1s2);
\draw[-latex,line width=2mm, color=black!40!white] (t1s2) -- (t1s3);
\draw[-latex,cap=round,line width=2mm, color=black!40!white] (t1s3) -- (t1t21.center) -- (t1t22.center) -- (t2s1);
\draw[-latex,line width=2mm, color=black!40!white] (t2s1) -- (t2s2);
\draw[-latex,line width=2mm, color=black!40!white] (t2s2) -- (t2s3);
\draw[-latex,cap=round,line width=2mm, color=black!40!white] (t2s3) -- (t2t31.center) -- (t2t32.center) -- (t3s1);
\draw[-latex,line width=2mm, color=black!40!white] (t3s1) -- (t3s2);
\draw[-latex,line width=2mm, color=black!40!white] (t3s2) -- (t3s3);

\draw(20,6) node[above] {\resizebox{2\width}{!}{\Huge \textbf{Initialization}}};
\draw(20,-7.5) node[above] {\resizebox{2\width}{!}{\Huge \textbf{Pruning}}};
% \draw(-1,-0) node[above,rotate=90] {\Huge Initialization}
\draw(-10,-14) node[above,rotate=90] {\resizebox{1.6\width}{!}{\Huge \textbf{Prune $\mathbf{t_1}$}}};
\draw(-10,-26) node[above,rotate=90] {\resizebox{1.6\width}{!}{\Huge \textbf{Prune $\mathbf{t_2}$}}};
\draw(-10,-38) node[above,rotate=90] {\resizebox{1.6\width}{!}{\Huge \textbf{Prune $\mathbf{t_3}$}}};
\end{tikzpicture}
\end{document}
