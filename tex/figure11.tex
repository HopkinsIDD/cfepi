% tikzpic.teP
\documentclass[crop,tikz]{standalone}% 'crop' is the default for v1.0, before it was 'preview'

%Packages
\usepackage{xcolor}

%tikz libraries}}
\usetikzlibrary{shapes.geometric,positioning}

%Colors
\definecolor{susceptible}{RGB}{141,160,203}
\definecolor{recovery}{RGB}{102,194,165}
\definecolor{infection}{RGB}{252,141,98}
% \definecolor{intervention}{RGB}{231,138,195}
\definecolor{intervention}{RGB}{50, 102, 86}
\definecolor{nothing}{RGB}{225,225,225}

%New commands
\newcommand{\Leftscissor}{\large  X}
\newcommand{\Largescissor}{\Huge  X}

% Formatting macros:
\tikzstyle{node}=[minimum size= .5in,circle, draw]
\tikzset{intervened node fill/.style  args={#1,#2}{node,potential,circle split part fill={#1,#2}}}

\tikzset{potential/.style={opacity=.7}}
\tikzset{actual/.style={opacity=1}}
\tikzset{impossible/.style={opacity=.2}}

\tikzstyle{edge}=[diamond, draw]
\tikzstyle{intervention}=[fill = intervention]
\tikzstyle{contact}=[fill = infection]
\tikzstyle{recovery}=[fill = recovery]

\makeatletter
\tikzset{circle split part fill/.style  args={#1,#2}{%
    alias=tmp@name, % Jake's idea !!
    postaction={%
      insert path={
        \pgfextra{%
          \pgfpointdiff{\pgfpointanchor{\pgf@node@name}{center}}%
          {\pgfpointanchor{\pgf@node@name}{north}}%
          \pgfmathsetmacro\scale{1}
          \pgfmathsetmacro\insiderad{\pgf@y*\scale}
          \fill[#1] (\pgf@node@name.base) ([yshift=-\pgflinewidth]\pgf@node@name.north) arc
          (90:270:\insiderad-\pgflinewidth)--cycle;
          \fill[#2] (\pgf@node@name.base) ([yshift=\pgflinewidth]\pgf@node@name.south)  arc
          (270:450:\insiderad-\pgflinewidth)--cycle;            %  \end{scope}
        }}}}}
 \makeatother  
 
 \makeatletter
\tikzset{circle tri split part fill/.style  args={#1,#2,#3}{%
    alias=tmp@name, % Jake's idea !!
    postaction={%
      insert path={
        \pgfextra{%
          \pgfpointdiff{\pgfpointanchor{\pgf@node@name}{center}}%
          {\pgfpointanchor{\pgf@node@name}{north}}%
          \pgfmathsetmacro\scale{1}
          \pgfmathsetmacro\insiderad{\pgf@y*\scale}
          \fill[#1] (\pgf@node@name.base) ([yshift=-\pgflinewidth]\pgf@node@name.north) arc
          (90:210:\insiderad-\pgflinewidth)--(\pgf@node@name.center)--cycle;
          \fill[#2] (\pgf@node@name.base) ([yshift=-\pgflinewidth]\pgf@node@name.north)  arc
          (90:-30:\insiderad-\pgflinewidth)--(\pgf@node@name.center)--cycle; 
          \fill[#3] (\pgf@node@name.base) ([yshift=\pgflinewidth]\pgf@node@name.south)  arc
          (-90:-30:\insiderad-\pgflinewidth)--(\pgf@node@name.center)--cycle; 
          \fill[#3] (\pgf@node@name.base) ([yshift=\pgflinewidth]\pgf@node@name.south)  arc
          (270:210:\insiderad-\pgflinewidth)--(\pgf@node@name.center)--cycle;            %  \end{scope}
        }}}}}
 \makeatother

\begin{document}
\begin{tikzpicture}
\draw (  0,  6) node             {\resizebox{1.6\width}{!}{\Huge \textbf{Potential Epidemic Graph}}};
\draw (  0,-0 ) node (potential) {\includegraphics{figure9_pg_1}};
\draw (-10,- 9) node (prei0)     {\resizebox{1.6\width}{!}{\Huge \textbf{Uncontrolled}}};
\draw (-10,-15) node (t0i0)      {\includegraphics{figure9_pg_2}};
\draw (-10,-27) node (t1i0)      {\includegraphics{figure9_pg_3}};
\draw (-10,-39) node (t2i0)      {\includegraphics{figure9_pg_4}};
\draw (-10,-51) node (t3i0)      {\includegraphics{figure9_pg_5}};
\draw ( 10,- 9) node (prei1)     {\resizebox{1.6\width}{!}{\Huge \textbf{Controlled}}};
\draw ( 10,-15) node (t0i1)      {\includegraphics{figure10_pg_2}};
\draw ( 10,-27) node (t1i1)      {\includegraphics{figure10_pg_3}};
\draw ( 10,-39) node (t2i1)      {\includegraphics{figure10_pg_4}};
\draw ( 10,-51) node (t3i1)      {\includegraphics{figure10_pg_5}};

\draw[-latex,line width=2mm, color=black!40!white] (potential) -- (prei0);
\draw[-latex,line width=2mm, color=black!40!white] (t0i0) -- (t1i0);
\draw[-latex,line width=2mm, color=black!40!white] (t1i0) -- (t2i0);
\draw[-latex,line width=2mm, color=black!40!white] (t2i0) -- (t3i0);

\draw[-latex,line width=2mm, color=black!40!white] (potential) -- (prei1);
\draw[-latex,line width=2mm, color=black!40!white] (t0i1) -- (t1i1);
\draw[-latex,line width=2mm, color=black!40!white] (t1i1) -- (t2i1);
\draw[-latex,line width=2mm, color=black!40!white] (t2i1) -- (t3i1);

\end{tikzpicture}
\end{document}
